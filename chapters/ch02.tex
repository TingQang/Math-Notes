% 章节文件:ch02.tex
% 测试README.md中定义的所有LaTeX功能(图、表、公式、交叉引用、代码、列表等)
\chapter{第二章:功能测试章节}
\label{ch:ch02}

\section{文本与基础格式测试}
\label{sec:text_format}

% 测试普通文本、加粗、斜体、脚注(README中常提及的基础格式)
本章用于测试README.md中定义的所有LaTeX功能,包括基础文本格式、图表插入、公式排版、交叉引用等。
加粗文本示例:\textbf{这是加粗文本};斜体文本示例:\textit{这是斜体文本};
脚注示例\footnote{这是测试脚注,验证脚注功能是否正常编译}。

\section{列表功能测试}
\label{sec:lists}

% 测试无序列表、有序列表(README中常见列表需求)
\subsection{无序列表}
\begin{itemize}
    \item 测试无序列表项1
    \item 测试无序列表项2
    \item 嵌套无序列表
          \begin{itemize}
              \item 嵌套项1
              \item 嵌套项2
          \end{itemize}
\end{itemize}

\subsection{有序列表}
\begin{enumerate}
    \item 测试有序列表项1
    \item 测试有序列表项2
    \item 嵌套有序列表
          \begin{enumerate}
              \item 嵌套项1
              \item 嵌套项2
          \end{enumerate}
\end{enumerate}

\section{公式功能测试}
\label{sec:equations}

% 测试行内公式、单行公式、多行公式(README中数学公式相关功能)
\subsection{行内公式}
行内公式示例:$E=mc^2$(质能方程),$f(x) = \sum_{i=1}^n x_i$(求和公式)。

\subsection{单行编号公式}
\begin{equation}
    \int_{-\infty}^{+\infty} e^{-x^2} dx = \sqrt{\pi}
    \label{eq:gaussian_integral}
\end{equation}
公式\eqref{eq:gaussian_integral}为高斯积分,验证公式交叉引用功能。

\subsection{多行公式}
\begin{align}
    \sin^2\theta + \cos^2\theta &= 1 \label{eq:trig1} \\
    \tan\theta &= \frac{\sin\theta}{\cos\theta} \label{eq:trig2}
\end{align}
公式\eqref{eq:trig1}和\eqref{eq:trig2}验证多行公式排版与引用。

\section{表格功能测试}
\label{sec:tables}

% 测试基础表格、带标题/引用的表格(README中表格相关功能)
\begin{table}[htbp]
    \centering
    \caption{测试表格(基础格式)}
    \label{tab:test_table}
    \begin{tabular}{|c|c|c|}
        \hline
        列1 & 列2 & 列3 \\
        \hline
        数据1 & 数据2 & 数据3 \\
        \hline
        数据4 & 数据5 & 数据6 \\
        \hline
    \end{tabular}
\end{table}

表格\ref{tab:test_table}验证表格插入、标题与交叉引用功能。

\section{图片功能测试}
\label{sec:figures}

% 测试图片插入、标题/引用、缩放(README中图相关核心功能)
% 统一图片路径为 ./figures/ch02/,符合LaTeX项目通用目录规范
\begin{figure}[htbp]
    \centering
    \includegraphics[width=0.6\textwidth]{ch02/three.pdf} 
    \caption{测试图片(验证图插入、缩放、标题)}
    \label{fig:test_figure}
\end{figure}

图\ref{fig:test_figure}%验证图片插入、缩放、标题与交叉引用功能(需确保test_image.png放在./figures/ch02/目录下)。

\section{代码块功能测试}
\label{sec:code_blocks}

% 测试LaTeX代码块排版(README中若提及代码展示功能)
\begin{verbatim}
# Python测试代码
def hello_world():
    print("Hello, LaTeX!")
hello_world()
\end{verbatim}

% 代码高亮配置(放在章节内,避免与主文档冲突)
\lstset{
    language=Python,
    basicstyle=\ttfamily\small,
    keywordstyle=\color{blue},
    commentstyle=\color{gray},
    stringstyle=\color{red},
    frame=single,
    breaklines=true
}

\begin{lstlisting}[caption={Python代码高亮示例}, label={lst:python_code}]
# 带高亮的Python测试代码
def add(a, b):
    """求和函数"""
    return a + b

result = add(3, 5)
print(f"结果:{result}")
\end{lstlisting}

代码块\ref{lst:python_code}验证代码高亮、标题与引用功能(需确保listings包已加载)。

\section{交叉引用综合测试}
\label{sec:cross_ref}

% 综合验证章节、公式、表格、图片、代码的交叉引用
\begin{itemize}
    \item 引用章节:第二章(\ref{ch:ch02})的\textsection\ref{sec:text_format};
    \item 引用公式:\eqref{eq:gaussian_integral}、\eqref{eq:trig1};
    \item 引用表格:\ref{tab:test_table};
    \item 引用图片:\ref{fig:test_figure};
    \item 引用代码:\ref{lst:python_code};
    \item 引用浮动体测试图:\ref{fig:float_test}。
\end{itemize}

\section{浮动体位置测试}
\label{sec:floats}

% 测试图表浮动体位置(README中常说明htbp等参数)
\begin{figure}[h] % h=here, t=top, b=bottom, p=page
    \centering
    \includegraphics[width=0.5\textwidth]{ch02/three.pdf}
    \caption{浮动体位置测试(h参数)}
    \label{fig:float_test}
\end{figure}

% 验证浮动体位置控制功能,需确保test_image2.png放在./figures/ch02/目录下。