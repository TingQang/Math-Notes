 %!TEX root = ../main.tex
\chapter{极限与连续}
\label{ch:limits}

\begin{center}
    \textcolor{kaoyan-blue}{\Large 本章学习目标:掌握极限的计算方法,理解连续性的本质}
\end{center}

\section{极限的概念与性质}

\subsection{数列极限}
\begin{definition}{数列极限}{}
    设$\{x_n\}$为数列,$a$为常数。若对于任意$\varepsilon>0$,存在正整数$N$,使得当$n>N$时,有$|x_n-a|<\varepsilon$,则称数列$\{x_n\}$收敛于$a$,记作$\lim\limits_{n\to\infty}x_n=a$。
\end{definition}

\begin{example}
    证明:$\lim\limits_{n\to\infty}\left(1+\frac{1}{n}\right)^n = e$
    \begin{solution}
        利用单调有界准则证明:
        \begin{enumerate}
            \item 先证数列$x_n = \left(1+\frac{1}{n}\right)^n$单调递增
            \item 再证数列有上界
            \item 由单调有界准则知极限存在,记该极限为$e$
        \end{enumerate}
    \end{solution}
\end{example}
\label{sec:sequence-limit-properties}
\keypoint{极限存在准则}:单调有界数列必收敛;夹逼准则。 

% 数列极限唯一性定理(用项目预设的theorem环境)
\begin{theorem}{唯一性}{thm:sequence-limit-uniqueness}
若数列$\{a_n\}$收敛,则它只有一个极限.
\end{theorem}

% 证明(用项目预设的proof环境,若未预设则用amsthm的proof)
\begin{proofbox}
设$a$是$\{a_n\}$的一个极限.我们证明:对任何数$b \neq a$,$b$不是$\{a_n\}$的极限.
事实上,若取$\varepsilon_0 = \frac{1}{2}|b - a|$,则按定义,在$U(a;\varepsilon_0)$之外至多只有$\{a_n\}$中有限项,从而在$U(b;\varepsilon_0)$内至多只有$\{a_n\}$中有限个项;所以$b$不是$\{a_n\}$的极限.
这就证明了收敛数列只能有一个极限.
\end{proofbox}
\begin{theorem}{有界性}{thm:sequence-limit-boundedness}
    若数列$\{a_n\}$收敛,则它是有界的。即存在正数$M$,使得对一切正整数$n$,都有
    \begin{equation}
        \absme{a_n} \leq M
    \end{equation}
\end{theorem}
\begin{proofbox}
    设$\lim\limits_{n\to\infty}a_n = a$,则对$\varepsilon = 1$,存在正整数$N$,使得当$n > N$时,有$\absme{a_N-a} < 1$。
    因此,对所有$n \geq 1$,都有
    $$
    \absme{a_n} = \absme{a_n - a + a} \leq \absme{a_n - a} + \absme{a} < 1 + \absme{a}
    $$
    令$M = \max\{\absme{a_1}, \absme{a_2}, \ldots, \absme{a_N}, 1 + \absme{a}\}$,则对一切正整数$n$都有$\absme{a_n} \leq M$。
\end{proofbox}
\subsection{函数极限}
\begin{definition}{函数极限}{}
    设函数$f(x)$在点$x_0$的某去心邻域内有定义,$A$为常数。若对于任意$\varepsilon>0$,存在$\delta>0$,使得当$0<|x-x_0|<\delta$时,有$|f(x)-A|<\varepsilon$,则称$f(x)$在$x\to x_0$时的极限为$A$。
\end{definition}

\formula{\lim_{x\to 0}\frac{\sin x}{x}=1}

\begin{attention}
    使用洛必达法则前必须检查是否满足$\frac{0}{0}$或$\frac{\infty}{\infty}$型未定式。
\end{attention}

\section{极限的计算方法}

\subsection{基本方法}
\begin{enumerate}
    \item \textbf{直接代入法}:初等函数在其定义域内连续
    \item \textbf{因式分解法}:消去零因子
    \item \textbf{有理化法}:处理根式
    \item \textbf{等价无穷小替换}:常用等价关系
\end{enumerate}

\subsection{洛必达法则}
\begin{theorem}{洛必达法则}{}
    若$\lim\limits_{x\to a}f(x)=\lim\limits_{x\to a}g(x)=0$(或$\infty$),且$\lim\limits_{x\to a}\frac{f'(x)}{g'(x)}$存在(或为$\infty$),则
    \[
    \lim_{x\to a}\frac{f(x)}{g(x)}=\lim_{x\to a}\frac{f'(x)}{g'(x)}
    \]
\end{theorem}

\begin{example}
    求$\lim\limits_{x\to 0}\frac{e^x-1-x}{x^2}$
    \begin{solution}
        这是$\frac{0}{0}$型未定式,使用洛必达法则:
        \begin{align*}
            \lim_{x\to 0}\frac{e^x-1-x}{x^2} 
            &= \lim_{x\to 0}\frac{e^x-1}{2x} \quad \text{(再次洛必达)}\\
            &= \lim_{x\to 0}\frac{e^x}{2} = \frac{1}{2}
        \end{align*}
    \end{solution}
\end{example}

\section{连续与间断}

\subsection{连续的定义}
\begin{definition}{连续}{}
    函数$f(x)$在点$x_0$连续,当且仅当:
    \begin{enumerate}
        \item $f(x_0)$存在
        \item $\lim\limits_{x\to x_0}f(x)$存在
        \item $\lim\limits_{x\to x_0}f(x)=f(x_0)$
    \end{enumerate}
\end{definition}

\subsection{间断点类型}
\begin{table}[ht]
    \centering
    \caption{间断点分类}
    \begin{tabular}{ccc}
        \toprule
        类型 & 条件 & 示例 \\
        \midrule
        可去间断点 & $\lim\limits_{x\to x_0}f(x)$存在但不等于$f(x_0)$ & $\frac{\sin x}{x} (x=0)$ \\
        跳跃间断点 & 左右极限存在但不相等 & $\operatorname{sgn}(x) (x=0)$ \\
        无穷间断点 & 极限为无穷大 & $\frac{1}{x} (x=0)$ \\
        振荡间断点 & 极限不存在且不为无穷 & $\sin\frac{1}{x} (x=0)$ \\
        \bottomrule
    \end{tabular}
\end{table}

\section{本章习题}

\begin{exercise}
    \begin{enumerate}
        \item 求极限:$\lim\limits_{x\to 0}\frac{\sqrt{1+x}-\sqrt{1-x}}{x}$
        \item 求极限:$\lim\limits_{x\to 0}\left(\frac{1}{x}-\frac{1}{e^x-1}\right)$
        \item 判断函数$f(x)=\frac{x^2-1}{x-1}$在$x=1$处的连续性
    \end{enumerate}
\end{exercise}

\section{本章总结}
\begin{summary}
    \begin{itemize}
        \item \important{极限的$\varepsilon$-$\delta$定义是理论基础}
        \item 掌握七大重要极限:
        \begin{enumerate}
            \item $\lim\limits_{x\to 0}\frac{\sin x}{x}=1$
            \item $\lim\limits_{x\to\infty}\left(1+\frac{1}{x}\right)^x=e$
            \item $\lim\limits_{x\to 0}(1+x)^{\frac{1}{x}}=e$
        \end{enumerate}
        \item 熟练运用洛必达法则\cite{同济高数}
        \item 理解连续\cite{浙大概率论,同济高数,浙大概率论,清华线性代数}与间断的实质\cite{考研数学分析}
    \end{itemize}
\end{summary}