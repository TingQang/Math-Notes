% ==============================================
% 考研数学笔记 - 完整宏包与自定义配置
% 文件:preamble.tex
% 作用:集中管理所有宏包、环境、自定义命令
% ==============================================

% ==================== 第一部分:编码与字体 ====================
% 注意:\documentclass 已在 main.tex 中定义,此处不要重复
\usepackage[UTF8]{ctex}                      % 中文支持
\usepackage{anyfontsize}                     % 任意字体大小
\usepackage[quiet]{fontspec}                 % 安静模式,减少警告

% ==================== 第二部分:页面布局 ====================
\usepackage[
    a4paper,
    left=2.5cm,
    right=2.5cm,
    top=2.5cm,
    bottom=2.5cm,
    headheight=15pt,
    footskip=1.2cm,
    marginparwidth=0cm,      % 移除边注,使页面更干净
    marginparsep=0cm
]{geometry}

% ==================== 第三部分:数学宏包 ====================
\usepackage{amsmath, amsthm, amssymb, amsfonts}  % AMS数学基础
\usepackage{mathtools}                           % 增强数学工具
\usepackage{bm}                                  % 粗体数学符号
\usepackage{cases}                               % 多行公式
\usepackage{siunitx}                             % 单位与数字

% \usepackage{physics}                             % 物理符号(包含很多数学符号)

% ==================== 第四部分:图形与颜色 ====================
\usepackage{graphicx}                    % 图片插入
\usepackage{tcolorbox}                   % 彩色框(含定理环境)
\tcbuselibrary{most, breakable, skins}   % 加载tcolorbox库
\usepackage{tikz}                        % 绘图
\usepackage{xcolor}                      % 颜色

% 定义考研专用颜色
\definecolor{kaoyan-blue}{RGB}{0, 102, 204}    % 主色调蓝色
\definecolor{kaoyan-red}{RGB}{204, 51, 0}      % 重点红色
\definecolor{kaoyan-green}{RGB}{0, 153, 0}     % 正确/答案绿色
\definecolor{kaoyan-orange}{RGB}{255, 153, 0}  % 警告/注意橙色
\definecolor{kaoyan-gray}{RGB}{245, 245, 245}  % 背景灰色

% ==================== 第五部分:排版与工具 ====================
\usepackage{enumitem}                    % 列表控制
\usepackage{booktabs}                    % 专业表格
\usepackage{multirow, multicol}          % 表格与多列
\usepackage{array}                       % 数组扩展
\usepackage{makecell}                    % 表格单元格
\usepackage{longtable}                   % 长表格

% ==================== 第六部分:页面样式 ====================
\usepackage{fancyhdr}                    % 页眉页脚
\usepackage{titlesec}                    % 标题格式
\usepackage{titletoc}                    % 目录格式
\usepackage{appendix}                    % 附录

% ==================== 第七部分:引用与链接 ====================
\usepackage{hyperref}                    % 超链接
\usepackage{cleveref}                    % 智能引用

% ==================== 第八部分:参考文献 ====================
\usepackage[backend=biber, style=gb7714-2015, gbpub=false]{biblatex}

% ==================== 第九部分:自定义环境(使用tcolorbox) ====================

% 1. 定理类环境
\newtcbtheorem[number within=chapter]{definition}{定义}{%
    colframe=kaoyan-blue, 
    colback=kaoyan-gray, 
    colbacktitle=kaoyan-blue!10, 
    coltitle=kaoyan-blue, 
    fonttitle=\bfseries, 
    boxed title style={size=small},
    breakable
}{def}

\newtcbtheorem[number within=chapter]{theorem}{定理}{%
    colframe=kaoyan-red, 
    colback=kaoyan-gray!30, 
    colbacktitle=kaoyan-red!10,
    coltitle=kaoyan-red, 
    fonttitle=\bfseries, 
    boxed title style={size=small},
    breakable
}{thm}

\newtcbtheorem[number within=chapter]{lemma}{引理}{%
    colframe=kaoyan-green, 
    colback=kaoyan-gray!30, 
    colbacktitle=kaoyan-green!10,
    coltitle=kaoyan-green, 
    fonttitle=\bfseries, 
    boxed title style={size=small},
    breakable
}{lem}

\newtcbtheorem[number within=chapter]{corollary}{推论}{%
    colframe=kaoyan-orange, 
    colback=kaoyan-gray!30, 
    colbacktitle=kaoyan-orange!10,
    coltitle=kaoyan-orange, 
    fonttitle=\bfseries, 
    boxed title style={size=small},
    breakable
}{cor}

\newtcbtheorem[number within=section]{remark}{注}{%
    colframe=gray, 
    colback=kaoyan-gray, 
    colbacktitle=gray!10,
    coltitle=gray, 
    fonttitle=\bfseries, 
    boxed title style={size=small},
    breakable
}{rem}

% 2. 例题与习题环境
\newtcolorbox{example}{
    colframe=kaoyan-green!70!black,
    colback=kaoyan-green!5,
    title=例题,
    fonttitle=\bfseries,
    breakable
}

\newtcolorbox{exercise}{
    colframe=kaoyan-blue!70!black,
    colback=kaoyan-blue!5,
    title=习题,
    fonttitle=\bfseries,
    breakable
}

\newtcolorbox{solution}{
    colframe=kaoyan-green!50!black,
    colback=kaoyan-green!3,
    title=解,
    fonttitle=\bfseries,
    breakable
}

\newtcolorbox{proofbox}{
    colframe=kaoyan-red!50!black,
    colback=kaoyan-red!3,
    title=证明,
    fonttitle=\bfseries,
    breakable
}

% 3. 总结与注意事项
\newtcolorbox{summary}{
    colframe=kaoyan-orange,
    colback=kaoyan-orange!5,
    title=本章总结,
    fonttitle=\bfseries,
    breakable
}

\newtcolorbox{attention}{
    colframe=kaoyan-red,
    colback=kaoyan-red!5,
    title=注意,
    fonttitle=\bfseries,
    breakable
}
% 4. 本章目标环境(适配现有配色/样式体系)
\newtcolorbox{chaptergoal}{
    colframe=kaoyan-blue!80!black,  % 边框色:考研蓝加深,和定义/习题呼应
    colback=kaoyan-blue!8,          % 背景色:浅蓝,视觉柔和
    title=本章目标,                 % 标题文字
    fonttitle=\bfseries\large,      % 标题字体:加粗+稍大,突出层级
    coltitle=kaoyan-blue!90!black,  % 标题文字色
    breakable,                      % 支持跨页
    left=8pt,                       % 左侧内边距,和其他环境统一
    right=8pt,
    top=5pt,
    bottom=5pt
}
% ==================== 第十部分:自定义命令(考研数学专用) ====================

% 1. 数学符号快捷命令
\newcommand{\R}{\mathbb{R}}          % 实数集
\newcommand{\N}{\mathbb{N}}          % 自然数集
\newcommand{\Z}{\mathbb{Z}}          % 整数集
\newcommand{\Q}{\mathbb{Q}}          % 有理数集
\newcommand{\C}{\mathbb{C}}          % 复数集
\renewcommand{\P}{\mathbb{P}}        % 概率
\newcommand{\E}{\mathbb{E}}          % 期望

% 2. 微分与导数
\newcommand{\ddme}{\mathrm{d}}         % 微分符号(直立体)
\newcommand{\pd}{\partial}           % 偏导符号
\newcommand{\pder}[2]{\frac{\partial #1}{\partial #2}}  % 偏导数
\newcommand{\gradme}{\nabla}           % 梯度
\newcommand{\diver}{\operatorname{div}} % 散度

% 3. 极限与积分
\newcommand{\limn}{\lim\limits_{n\to\infty}}
\newcommand{\limx}{\lim\limits_{x\to\infty}}
\newcommand{\limxa}{\lim\limits_{x\to a}}
\newcommand{\intab}{\int_{a}^{b}}
\newcommand{\iintD}{\iint\limits_{D}}
\newcommand{\ointC}{\oint\limits_{C}}

% 4. 矩阵与向量
\newcommand{\mat}[1]{\begin{pmatrix} #1 \end{pmatrix}}      % 圆括号矩阵
\newcommand{\bmat}[1]{\begin{bmatrix} #1 \end{bmatrix}}     % 方括号矩阵
\newcommand{\absme}[1]{\left| #1 \right|}                     % 绝对值
\newcommand{\normme}[1]{\left\| #1 \right\|}                  % 范数

% 5. 概率与统计
\newcommand{\Var}{\operatorname{Var}}    % 方差
\newcommand{\Cov}{\operatorname{Cov}}    % 协方差

% 6. 快捷输入
\newcommand{\dx}{\,\mathrm{d}x}          % 积分微元
\newcommand{\dy}{\,\mathrm{d}y}
\newcommand{\dt}{\,\mathrm{d}t}
\newcommand{\const}{\text{const}}        % 常数

% 7. 考点标记
\newcommand{\keypoint}[1]{\textbf{\textcolor{kaoyan-red}{【考点】#1}}}
\newcommand{\important}[1]{\textbf{\textcolor{kaoyan-blue}{【重点】#1}}}
\newcommand{\difficulty}[1]{\textbf{\textcolor{kaoyan-orange}{【难度】#1}}}
\newcommand{\formula}[1]{\textcolor{kaoyan-blue}{\[#1\]}}

% ==================== 第十一部分:页面样式设置 ====================

% 页眉页脚设置
\pagestyle{fancy}
\fancyhf{}
\fancyhead[L]{\leftmark}
\fancyhead[R]{\rightmark}
\fancyfoot[C]{\thepage}
\renewcommand{\headrulewidth}{0.4pt}
\renewcommand{\footrulewidth}{0pt}

% 章节格式设置
\titleformat{\chapter}[display]
    {\normalfont\huge\bfseries\color{kaoyan-blue}}
    {\chaptertitlename\ \thechapter}{20pt}{\Huge}
\titlespacing*{\chapter}{0pt}{-30pt}{40pt}

% 节格式设置
\titleformat{\section}
    {\normalfont\Large\bfseries\color{kaoyan-blue}}
    {\thesection}{1em}{}
\titlespacing*{\section}{0pt}{3.5ex plus 1ex minus .2ex}{2.3ex plus .2ex}

% 超链接设置
\hypersetup{
    colorlinks=true,
    linkcolor=kaoyan-blue,
    citecolor=kaoyan-green,
    filecolor=magenta,
    urlcolor=kaoyan-blue,
    bookmarksopen=true,
    pdftitle={考研数学笔记},
    pdfauthor={你的名字},
    pdfsubject={考研数学复习资料}
}

% 列表格式设置
\setlist[enumerate]{
    topsep=0pt,
    itemsep=0pt,
    parsep=0pt,
    label=\arabic*),
    leftmargin=2em
}

\setlist[itemize]{
    topsep=0pt,
    itemsep=0pt,
    parsep=0pt,
    leftmargin=2em
}

% 表格设置
\setlength{\heavyrulewidth}{0.12em}
\setlength{\lightrulewidth}{0.05em}
\renewcommand{\arraystretch}{1.2}

% ==================== 第十二部分:计数器设置 ====================

% 公式编号带章节号
\numberwithin{equation}{chapter}

% 图片表格编号带章节号
\renewcommand{\thefigure}{\thechapter-\arabic{figure}}
\renewcommand{\thetable}{\thechapter-\arabic{table}}

% 重置计数器
\AtBeginDocument{
    \setcounter{chapter}{0}
    \setcounter{section}{0}
}